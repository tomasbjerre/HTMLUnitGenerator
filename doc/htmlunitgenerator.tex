\documentclass[a4paper,11pt]{kth-mag}
\usepackage[T1]{fontenc}
\usepackage{textcomp}
\usepackage{lmodern}
\usepackage[latin1]{inputenc}
\usepackage[swedish,english]{babel}
\usepackage{listings}
\usepackage{modifications}
\title{HTMLUnitGenerator}
\subtitle{Enables user friendly and powerful front end testing of web applications with minimum required effort to implement.}
\author{Author: Tomas Bjerre, tomasbjerre [AT] yahoo.com}
\blurb{Updated \today}
\begin{document}
\frontmatter
\pagestyle{empty}
\removepagenumbers
\maketitle

\tableofcontents*
\mainmatter

\pagestyle{newchap}
\chapter{Introduction}
This is the official documentation of HTMLUnitGenerator. This introduction includes three questions and three answers. They are intended to quickly get you into the context of this document.

\section{What is HTMLUnitGenerator?}
%A DSL
It is a compiler. It translates a user friendly DSL\footnote{Domain Specific Language} into a more advanced test case. That way you get clear test cases that are easily maintained while at the same time powerful and easily introduced in your current test suit.

\section{Why should I use HTMLUnitGenerator?}
\begin{enumerate}
\item \textbf{Easy to learn and fast to work with}\\
%Easy to understand
The time you need to spend reading up on HTMLUnitGenerator before being able to produce qualitative test cases using it, is very short.

%The alternative is to create raw code.
If you write your test cases using, for example, raw Java and HTMLUnit you will need to come up with some hierarchy of classes to be able to re use code. XPath:s and URL:s should typically be defined once and then referenced in all your test cases. Developing such a structure takes time as well as explaining and documenting it to your colleges.

\item \textbf{No need to document test cases}\\
%Self documenting
The Flow language (see Section \ref{flowlanguage}) is simple enough to, itself, qualify as documentation. Anyone, even people with no previous programming experience, will understand what your test cases do. The Flow language has been designed with the intention to provide only an absolute minimum number of choices to the developer, in order to keep all test cases clean and neutral.

\item \textbf{Future safe}\\
%Open Source, will be able to stop using it at any time
HTMLUnitGenerator is an open source software. You can write your own generator if you don't want to use HTMLUnit anymore. Or maybe you want to test your code using several different browsers. The idea is to provide several different generators with HTMLUnitGenerator\footnote{Yes, the name will change!}.
\end{enumerate}

\chapter{Flow language}
\label{flowlanguage}
This chapter will describe the Flow language. It is the language used to express test cases.

This chapter starts with a quick look at a test case written in Flow, see Section \ref{flowintro}, and continued with a complete walk through of the language. 

\section{Introduction}
\label{flowintro}
A quick example of a test case written in Flow is presented in Listing \ref{listinghrefofatag}. The details of this test case is explained later in this chapter.

% A small example
\lstset{basicstyle=\footnotesize, caption=Find href of an a tag, label=listinghrefofatag, numbers=left, frame=single, captionpos=b}
\lstinputlisting{../compiler/frontend/flowspec/testcases/simple.flow}

\section{Defining paths}
%what is XPath?
Flow uses XPath\footnote{http://www.w3schools.com/xpath/} to define containers, like \textit{div}-tags for example.

\lstset{basicstyle=\footnotesize, caption=Defining XPaths, label=listingpaths, numbers=left, frame=single, captionpos=b}
\lstinputlisting{../compiler/frontend/flowspec/testcases/paths.flow}

In Listing \ref{listingpaths} a \textit{div} is defined as \textit{searchpopup}, an element with an \textit{id} is defined as \textit{search} and the entire website content is defined as \textit{website}.

\section{Defining URL:s}
\label{flowurls}
%just show how to define them
URL:s are used in Flow to go to exact URL:s, see Section \ref{flowgoto}.

\lstset{basicstyle=\footnotesize, caption=Defining URLs, label=urls, numbers=left, frame=single, captionpos=b}
\lstinputlisting{../compiler/frontend/flowspec/testcases/urls.flow}

In Listing \ref{urls} a URL is defined as \textit{myUrl} and another defined as \textit{myOtherUrl}.

\section{Go to URL:s}
\label{flowgoto}
%Go to
The \textit{Go to} statement is used to browse directly to a URL, see Section \ref{flowurls}.

\lstset{basicstyle=\footnotesize, caption=Go to statement, label=gotosimple, numbers=left, frame=single, captionpos=b}
\lstinputlisting{../compiler/frontend/flowspec/testcases/gotosimple.flow}

In Listing \ref{gotosimple} a URL is defined and used in a \textit{Go to} statement. This will create a test case where a browser is opened and the URL \textit{http://www.my.domain/page.html} is visited.


So far, no assertions on the content of the visited page are made. This means the test will never fail. In order to make assertions you may want to wait for a while after the page has been loaded, in order for JavaScripts to execute. This can be done as in Listing \ref{gotowait}.

%Go to and wait x seconds
\lstset{basicstyle=\footnotesize, caption=Go to statement with wait, label=gotowait, numbers=left, frame=single, captionpos=b}
\lstinputlisting{../compiler/frontend/flowspec/testcases/gotowait.flow}

%Refer to Find, if you want timeout
In Listing \ref{gotowait} the test case will sleep for 5 seconds before continuing execution. This may work in many situations but it should be avoided. You should instead wait for a condition to be true and then continue execution. This can be achieved with \textit{wait at most} in the \textit{Find} statement (see Section \ref{findwaitatmost}).


\section{Find elements}
%containing
The \textit{Find} statement is used to make assertions on the content of the current page.

\lstset{basicstyle=\footnotesize, caption=Find statement, label=findsimple, numbers=left, frame=single, captionpos=b}
\lstinputlisting{../compiler/frontend/flowspec/testcases/findsimple.flow}

In Listing \ref{findsimple} a web page is visited and the test will fail unless it contains ``\textit{Hello}''.

%with spaces and quotes, or no spaces and no quotes, or quotes and no spaces
If you want to match a string containing white spaces you must add quotes to the string, see Listing \ref{findstring}.

\lstset{basicstyle=\footnotesize, caption=Find statement with spaces, label=findstring, numbers=left, frame=single, captionpos=b}
\lstinputlisting{../compiler/frontend/flowspec/testcases/findstring.flow}

%with in statement
By default \textit{Find} will look for the given content within the entire website, \textit{/html/body}. You may change this behaviour by adding \textit{in [XPath]}. Listing \ref{findstringin} shows how the string \textit{Hello World} is asserted to be found in inside an element with \textit{id} set to \textit{mypopup}. 

\lstset{basicstyle=\footnotesize, caption=Find statement in given XPath, label=findstringin, numbers=left, frame=single, captionpos=b}
\lstinputlisting{../compiler/frontend/flowspec/testcases/findstringin.flow}

%several containing
Boolean logic, \textit{and}/\textit{or}, can be used to accept different combinations of containing strings. Listing \ref{findstringandor} shows how the string \textit{Hello World} is asserted to be found along with \textit{Alternative1} or \textit{Alternative2}.

\lstset{basicstyle=\footnotesize, caption=Find statement with boolean logics, label=findstringandor, numbers=left, frame=single, captionpos=b, breaklines=true}
\lstinputlisting{../compiler/frontend/flowspec/testcases/findstringandor.flow}

%tag
Find can also be used to find tags. In Listing \ref{findtag} an assertion is made making sure an \textit{a} tag with attribute \textit{href} set to \textit{/link/category/blandat} is available.

\lstset{basicstyle=\footnotesize, caption=Find tag statement, label=findtag, numbers=left, frame=single, captionpos=b, breaklines=true}
\lstinputlisting{../compiler/frontend/flowspec/testcases/findtag.flow}

%tag in
In Listing \ref{findtagin} the a tag is asserted to be available within the XPath \textit{mypopup}.

\lstset{basicstyle=\footnotesize, caption=Find tag statement within an XPath, label=findtagin, numbers=left, frame=single, captionpos=b, breaklines=true}
\lstinputlisting{../compiler/frontend/flowspec/testcases/findtagin.flow}

%several tags
Tags can also be combined, as in Listing \ref{findtagandor}.

\lstset{basicstyle=\footnotesize, caption=Find tag statement with and/or logics, label=findtagandor, numbers=left, frame=single, captionpos=b, breaklines=true}
\lstinputlisting{../compiler/frontend/flowspec/testcases/findtagandor.flow}

%several attributes
Several attributes can be asserted to be available in the same tag. Listing \ref{findtagattributes} shows how tag \textit{a} is asserted to be available with either \textit{/link/category/mix} or \textit{/link/category/other} but then attribute \textit{class} has to be defined as \textit{otherClass}.

\lstset{basicstyle=\footnotesize, caption=Find tag with several attributes, label=findtagattributes, numbers=left, frame=single, captionpos=b, breaklines=true}
\lstinputlisting{../compiler/frontend/flowspec/testcases/findtagattributes.flow}

%Combine tags and strings
Find can combine tags and strings. Listing \ref{findtagandorcontaining} shows how \textit{a} with attribute \textit{href} set to \textit{/link/category/mix} along with string \textit{Alternative1} or \textit{Alternative2} is asserted to be available.

\lstset{basicstyle=\footnotesize, caption=Find statement with combined tag and containing with and/or logics, label=findtagandorcontaining, numbers=left, frame=single, captionpos=b, breaklines=true}
\lstinputlisting{../compiler/frontend/flowspec/testcases/findtagandorcontaining.flow}

%Show time out example
\label{findwaitatmost}
The content asserted by \textit{Find} may not be available, and then the test will fail. You may accept the content to be available within a period of time, for example JavaScripts may update the DOM. Listing \ref{findwait} shows how \textit{wait at most} is used to wait at most 5 seconds for the assertion to become true.

\lstset{basicstyle=\footnotesize, caption=Find statement with wait at most, label=findwait, numbers=left, frame=single, captionpos=b, breaklines=true}
\lstinputlisting{../compiler/frontend/flowspec/testcases/findwait.flow}

\section{Click on elements}
%can be used for dynamic html or page navigation
A click on an element may result in the entire page being reloaded or just a JavaScript adding some content to the DOM. Listing \ref{clickon} shows a simple example of an element being clicked.

\lstset{basicstyle=\footnotesize, caption=Click on, label=clickon, numbers=left, frame=single, captionpos=b, breaklines=true}
\lstinputlisting{../compiler/frontend/flowspec/testcases/clickon.flow}

%and wait
After clicking the element you will probably want to add a \textit{Find} statement. Listing \ref{clickonwait} show how a waiting period can be added after the click. When waiting for content to be available it is recommended to use \textit{Find} with \textit{wait at most} as in Section \ref{findwaitatmost}.

\lstset{basicstyle=\footnotesize, caption=Click on and wait, label=clickonwait, numbers=left, frame=single, captionpos=b, breaklines=true}
\lstinputlisting{../compiler/frontend/flowspec/testcases/clickonwait.flow}

\section{Fill in forms}
%Fill in one element
When testing pages that contains forms you may need to add text to them. Listing \ref{fillinsimple} shows how a text field can be populated with content. The URL \textit{http://someurl.domain/} is asserted to contain a form named \textit{userForm} which contains a text field named \textit{name}. The text field is populated with ``\textit{Tomas Bjerre}''.

\lstset{basicstyle=\footnotesize, caption=Fill in text field in formular, label=fillinsimple, numbers=left, frame=single, captionpos=b, breaklines=true}
\lstinputlisting{../compiler/frontend/flowspec/testcases/fillinsimple.flow}

%Fill in multiple element
Listing \ref{fillinmulti} shows how multiple text fields can be populated.

\lstset{basicstyle=\footnotesize, caption=Fill in multiple text fields in formular, label=fillinmulti, numbers=left, frame=single, captionpos=b, breaklines=true}
\lstinputlisting{../compiler/frontend/flowspec/testcases/fillinmulti.flow}

%Fill in with unique string
Listing \ref{fillinunique} shows how the text field \textit{name} is populated with a unique (random) string.

\lstset{basicstyle=\footnotesize, caption=Fill in text field in formular with unique string, label=fillinunique, numbers=left, frame=single, captionpos=b, breaklines=true}
\lstinputlisting{../compiler/frontend/flowspec/testcases/fillinunique.flow}

%Fill in with unique string of length
The string in Listing \ref{fillinunique} can be set to a specific length as in Listing \ref{fillinuniquelength}.

\lstset{basicstyle=\footnotesize, caption=Fill in text field in formular with unique string of specific length, label=fillinuniquelength, numbers=left, frame=single, captionpos=b, breaklines=true}
\lstinputlisting{../compiler/frontend/flowspec/testcases/fillinuniquelength.flow}

%Fill in with unique string of length starting with
The string in Listing \ref{fillinuniquelength} may only be partly random as in Listing \ref{fillinuniquelengthstart}. In Listing \ref{fillinuniquelengthstart} the string is randomly generated but it will always start with the same sequence.

\lstset{basicstyle=\footnotesize, caption=Fill in text field in formular with unique string of specific length and start, label=fillinuniquelengthstart, numbers=left, frame=single, captionpos=b, breaklines=true}
\lstinputlisting{../compiler/frontend/flowspec/testcases/fillinuniquelengthstart.flow}

%Fill in locationForm with address.floor as 3
%Fill in locationForm with address.floor as options number 1
When selecting an item from a drop down, the same code as above can be used. But it may be preferred to reference the item as the index number. Listing \ref{fillinoption} shows how the option number 1 is selected.

\lstset{basicstyle=\footnotesize, caption=Fill in option by index number, label=fillinoption, numbers=left, frame=single, captionpos=b, breaklines=true}
\lstinputlisting{../compiler/frontend/flowspec/testcases/fillinoption.flow}

\section{Re-use test cases}
%See PathsAndUrls.flow
It is ofter preferred to store definitions of XPath:s and URL:s in separate files. Then they can be re-used in test cases. Also there may be a serious of actions needed in order to get to a specific state, such actions may also be stored in separate files and enabled to be re-used. Listing \ref{see} shows how test code can be made available in another test.

\lstset{basicstyle=\footnotesize, caption=See statement, label=see, numbers=left, frame=single, captionpos=b, breaklines=true}
\lstinputlisting{../compiler/frontend/flowspec/testcases/see.flow}

\section{Use proxy}
Test cases may need to be configured to use proxies when run. Listing \ref{useproxy} shows how to set the proxy server to use.

\lstset{basicstyle=\footnotesize, caption=Use proxy statement, label=useproxy, numbers=left, frame=single, captionpos=b, breaklines=true}
\lstinputlisting{../compiler/frontend/flowspec/testcases/useproxy.flow}

\end{document}
